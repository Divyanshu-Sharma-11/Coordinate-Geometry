\documentclass[12pt]{article}
\usepackage{amsmath}
\newcommand{\myvec}[1]{\ensuremath{\begin{pmatrix}#1\end{pmatrix}}}
\newcommand{\mydet}[1]{\ensuremath{\begin{vmatrix}#1\end{vmatrix}}}
\newcommand{\solution}{\noindent \textbf{Solution: }}
\providecommand{\brak}[1]{\ensuremath{\left(#1\right)}}
\providecommand{\norm}[1]{\left\lVert#1\right\rVert}
\let\vec\mathbf

\title{Coordinate Geometry}
\author{Divyanshu Sharma (divyansusharma@sriprakashschools.com)}

\begin{document}
\maketitle
\section*{10$^{th}$ Maths - Chapter 7}
This is Problem-1(ii) from Exercise 7.3
\begin{enumerate}
\item Find the area of the triangle whose vertices are : 
\\ (ii) (–5, –1), (3, –5), (5, 2)
\\\solution \\ 
Given Data: 
\\x1 =-5, 
\\x2 = 3, 
\\x3 = 5, 
\\y1 = -1, 
\\y2 = -5 
\\y3 = 2
\begin{align}
\\Area of the triangle &=\frac{1}{2} [-5 { (-5-2)} + 3(2+1) + 5{(-1+5)}]
\\ &=frac{1}{2}{(35 + 9 + 20)} \\ &= 32
\end{align}
\\Therefore, the area of the triangle is 32 square units.


\end{enumerate}



\end{document}
